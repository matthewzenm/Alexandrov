\chapter{General Topology}

\section{Topology and Continuity}

In this section, we start our tour by introducing basic topology conepts.

\begin{defn}
    Let $X$ be a set.
    A \emph{topology} $\mathcal{T}$ on $X$ is a collection of subsets of $X$ satisfies
    \begin{enumerate}[(1)]
        \item $\mathcal{T}$ contains $\varnothing$ and $X$;
        \item For any collection $\{U_i\}_{i\in I}\subset\mathcal{T}$, we have $\bigcup_{i\in I}U_i\in\mathcal{T}$;
        \item For $U,V\in\mathcal{T}$, we have $U\cap V\in\mathcal{T}$.
    \end{enumerate}
    A set in $\mathcal{T}$ is called an \emph{open set}.
    A set together with a topology on it is called a \emph{topological space}.
\end{defn}

\begin{rem}
    By induction, being open is stable under finite intersection, that is, for any finite many of open sets $U_1,\cdots,U_n$, we have $\bigcup_{i=1}^nU_i\in\mathcal{T}$.
\end{rem}

\begin{defn}
    If $\mathcal{T}_1,\mathcal{T}_2$ are two topologies on $X$, we say that $\mathcal{T}_1$ is coarser than $\mathcal{T}_2$ (or equivalently $\mathcal{T}_2$ is finer than $\mathcal{T}_1$) if every open set in $\mathcal{T}_1$ is an open set in $\mathcal{T}_2$.
    Clearly this is a partial order. 
\end{defn}

\begin{defn}
    Let $\mathcal{B}=\{B_i\}_{i\in I}$ is a collection of subsets of a topological space $X$.
    If any open set of $X$ is a union of elements in $\mathcal{B}$, then $\mathcal{B}$ is called a \emph{basis} of $X$, and we say $\mathcal{B}$ generates (the topology of) $X$.
\end{defn}

\begin{defn}
    A \emph{continuous mapping}, or simply a \emph{map}, $f:X\to Y$ between topological spaces $X,Y$ is a mapping such that for any open set $U\subset Y$, the inverse image $f^{-1}(U)\subset X$ is an open set.
\end{defn}

\begin{lem}\label{lem:category}
    Identity is continuous.
    The composition of continuous mappings is continuous.
\end{lem}
\begin{proof}
    Obvious.
\end{proof}

\begin{defn}
    Let $f:X\to Y$, $g:Y\to X$ are maps between $X,Y$, if $g\circ f=\mathrm{id}_X$ and $f\circ g=\mathrm{id}_Y$, then $X$ and $Y$ are called to be \emph{homeomorphic}.
    $f,g$ are called \emph{homeomorphisms}.
\end{defn}

\begin{eg}\label{eg:product space}
    Let $X,Y$ be two topological spaces.
    The \emph{product space} of $X,Y$, denoted by $X\times Y$, is the topological spaces on the Cartesian product $X\times Y$ (by abuse of notation), whose topology is generated by
    \[\{U\times V\colon U,V\text{ are open in }X,Y\text{ respectively}\}.\]
    Product spaces are characterized as follows:
    If $Z$ is a topological space with maps $p_X:Z\to X$, $p_Y:Z\to Y$, and have the property that for any topological space $Z'$ with maps $f_X:Z'\to X$, $f_Y:Z'\to Y$, there exists a unique map $f:Z'\to Z$ such that the following diagram commutes
    \[\begin{tikzcd}
        Z' \ar[drr, bend left, "{f_X}"] \ar[ddr, bend right, "{f_Y}"] \ar[dr, dashed, "f"] & \ & \ \\
        \ & Z \ar[r, "{p_X}"] \ar[d, "{p_Y}"] & X \\
        \ & Y, & \
    \end{tikzcd}\]
    then $Z$ is homeomorphic to $X\times Y$.
\end{eg}

Please prove the claim.

\begin{eg}\label{eg:subspace}
    Let $A\subset X$ is a subset of topological space $X$.
    Then $A$ carries a \emph{subspace topology} naturally:
    Let $U\subset A$ be open if and only if $U=A\cap\tilde{U}$ for some $\tilde{U}$ open with respective to $X$, then this is a topology on $A$.
\end{eg}

Please prove this claim.

\begin{rem}
    If you know some category theory, \Cref{lem:category} says that topological spaces together with continuous mappings consists a category, \Cref{eg:product space} gives the construction of products in the category of topological spaces.
\end{rem}

\section{Open and Closed Sets}

\begin{defn}
    Let $X$ be a topological space, $Z\subset X$ is a subset.
    $Z$ is said to be \emph{closed} if $X-Z$ is open.
\end{defn}

\begin{defn}
    Let $X$ be a topological space, $S\subset X$ and $x\in X$.
    If there exists an open set $U\ni x$ such that $U\subset S$, then $x$ is called an \emph{interior point} of $S$.
    If for any open set $U\ni x$, $U-\{x\}$ contains a point of $S$, then $x$ is called an \emph{accumulation point} of $S$.
\end{defn}

\begin{prop}
    Let $X$ be a topological space and $S\subset X$.
    \begin{enumerate}[(1)]
        \item $S$ is open if and only if every point of $S$ is an interior point.
        \item $S$ is closed if and only if every accumulation point of $S$ is in $S$.
    \end{enumerate}
\end{prop}

Please prove this proposition.

\begin{defn}
    Let $X$ be a topological space and $x\in X$.
    A \emph{neighborhood} of $x$ is a subset of $X$ such that $x$ is an interior point of it.
\end{defn}

\begin{defn}
    Let $X$ be a topological space and $S\subset X$.
    \begin{enumerate}[(1)]
        \item The \emph{interior} of $S$ is the maximal open set that contained in $S$.
        \item The \emph{closure} of $S$ is the minimal closed set that contains $S$.
    \end{enumerate}
\end{defn}

\begin{rem}
    The closure of a nonempty subset is always nonempty, but its interior can be empty.
\end{rem}

\section{Countable and Separable Axioms}

We introduce two most important axioms in general topology.

\begin{defn}
    A topological space is called to satisfy \emph{second countable axiom} (C2 axiom) if it has a countable basis.
\end{defn}

\begin{defn}
    A topological space is called to satisfy \emph{Hausdorff axiom} (T2 axiom) if for any $p,q$, there are neighborhoods $U,V$ of $p,q$ such that $U\cap V=\varnothing$.
\end{defn}

\begin{prop}
    Topological space $X$ is Hausdorff if and only if the diagonal $\Delta=\{(x,x)\in X\times X\colon x\in X\}$ of product space $X\times X$ is closed.
\end{prop}
\begin{proof}
    If $X$ is Hausdorff, we show that $X\times X-\Delta$ is open.
    Let $(x,y)$ with $x\neq y$ be any point in $X\times X-\Delta$, since $X$ is Hausdorff, there exists open sets $U\ni x$, $V\ni y$ with $U\cap V=\varnothing$.
    Then $U\times V$ is an open set in $X\times X$, and since $U\cap V=\varnothing$, we have $U\times V\cap\Delta=\varnothing$.
    Thus $(x,y)\in U\times V\subset X\times X-\Delta$ is an interior point, and therefore $X\times X-\Delta$ is open.
    Conversely, if $\Delta$ is closed, let $x\neq y\in X$, then $(x,y)\in X\times X-\Delta$.
    Thus there exists an open set in $X\times X-\Delta$ that contains $(x,y)$, and since $X\times X$ is generated by $\{U\times V\colon U,V\text{ open}\}$, $(x,y)\in U\times V$ for some open $U,V$.
    Moreover, since $U\times V\subset X\times X-\Delta$, we have $U\cap V=\varnothing$.
\end{proof}

\begin{prop}
    The product of two Hausdorff spaces is Hausdorff.
\end{prop}

Please prove this claim.

\section{Compact Spaces}

\begin{defn}
    An \emph{open cover} of a topological space $X$ is a collection of open sets $\mathcal{U}=\{U_i\}_{i\in I}$ such that $X=\bigcup_{i\in I}U_i$.
    A (open) \emph{subcover} of $\mathcal{U}$ is a subset of $\mathcal{U}$ that is still an open cover.
\end{defn}

\begin{defn}
    A topological space $X$ is called a \emph{compact space} if any open cover of $X$ has a finite subcover.
\end{defn}

Similarly we have the notion of open covers of a set and compact subsets.
Please give these definitions (or you can deduce them from the following lemma).

\begin{lem}
    Let $K\subset X$, then $K$ is a compact space with respective to subspace topology if and only if $K$ is a compact subset of $X$.
\end{lem}
\begin{proof}
    Suppose $K$ is a compact space with respective to subspace topology, then for any open cover $\{\tilde{U}_i\}_{i\in I}$ in $X$ of $K$, $\{U_i=\tilde{U}_i\}_{i\in I}$ is an open cover of topological space $K$.
    $\{U_i\}$ has a finite subcover, say $U_1,\cdots,U_n$, then $\tilde{U}_1,\cdots,\tilde{U}_n$ is a subcover of $\{\tilde{U}_i\}_{i\in I}$.
    Conversely, suppose $K$ is a compact subset in $X$, reverse the argument above we can prove $K$ is a compact space with respective to subspace topology.
\end{proof}

We list some important properties of compact spaces.

\begin{prop}
    Let $X$ be a topological space, $K\subset X$ be a compact subset, $f:X\to Y$ is a continuous mapping.
    Then $f(K)$ is compact in $Y$.
\end{prop}
\begin{proof}
    Let $\{U_i\}_{i\in I}$ be an open cover of $f(K)$ in $Y$, then $\{f^{-1}(U_i)\}_{i\in I}$ is an open cover of $f^{-1}(f(K))\supset K$, has a finite subcover $f^{-1}(U_1),\cdots,f^{-1}(U_n)$.
    Then $U_1,\cdots,U_n$ is a finite subcover of $f(K)$.
\end{proof}

\begin{prop}
    A closed subset of a compact space is compact.
\end{prop}
\begin{proof}
    Let $X$ be a compact space and $Z$ be a closed subset.
    If $\mathcal{U}$ is an open cover of $Z$, then $\mathcal{U}\cup\{X-Z\}$ is an open cover of $X$, which has a finite subcover.
    Exclude $X-Z$ if necessary, we obtain a subcover of $\mathcal{U}$ of $Z$.
\end{proof}

\begin{prop}
    A compact subset of a Hausdorff space is closed.
\end{prop}
\begin{proof}
    Let $X$ be a Hausdorff space, $K$ be a compact subset.
    Let $x\notin K$, we prove $x$ is an interior point of $X-K$.
    For any $k\in K$, we associate $k$ with two open sets $U_k\ni k$ and $V_k\ni x$ with $U_k\cap V_k=\varnothing$, this can be achieved since $X$ is Hausdorff.
    Then $\{U_k\}_{k\in K}$ is an open cover of $K$.
    Since $K$ is compact, it has a subcover $\{U_{k_i}\}_{1\leq i\leq n}$.
    Then $V=\bigcap_{i=1}^nV_{k_i}$ is disjoint with $K$, and is open.
    Thus $x$ is an interior point of $X-K$, $X-K$ is open, and consequencely $K$ is closed.
\end{proof}

\begin{cor}
    A subset of a compact Hausdorff space is compact if and only if its closed.
\end{cor}

\begin{defn}
    A topological space is called \emph{locally compact} if any point has a compact neighborhood.
\end{defn}

\section{Connected and Path Connected Spaces}

The title of this section indicates two types of connectedness.
The first one is more general.

\begin{defn}
    A topological space $X$ is \emph{disconnected} if $X=U\cup V$ for some nonempty open $U,V$ such that $U\cap V=\varnothing$.
    A subset $A\subset X$ is \emph{disconnected} if $A$ is disconnected with respective to the subspace topology.
    A space or a subset is \emph{connected} if it is not disconnected.
\end{defn}

\begin{lem}\label{lem:connected component}
    Let $X$ be a topological space and $x\in X$, and $\{C_i\}_{i\in I}$ is the collection of all connected subsets contains $x$.
    Set $C=\bigcup_{i\in I}C_i$, then $C$ is connected.
\end{lem}
\begin{proof}
    Suppose $C=U\cup V$ for some disjoint open sets $U,V$ with respective to $C$, assume without loss of generality that $x\in U$.
    For each $i\in I$, $C_i=(C_i\cap U)\cup(C_i\cap V)$, then $C_i\cap U$ and $C_i\cap V$ are disjoint.
    But $C_i$ is connected, $C_i\cap V$ must be empty, since $x\in C_i\cap U$.
    Thus
    \[V=C\cap V=\bigcup_{i\in I}(C_i\cap V)=\varnothing,\]
    $C$ is connected.
\end{proof}

\begin{defn}
    According to \Cref{lem:connected component}, the union of all connected subsets that contains a point $x$ is connected, it is called the \emph{connected component} which $x$ lies in.
\end{defn}

\begin{prop}
    Two connected components are either disjoint or coincide.
    Thus a topological space is partitioned into connected components.
\end{prop}

Please prove this claim.

\begin{prop}\label{prop:connected image}
    Let $X,Y$ be topological spaces and $X$ is connected, $f:X\to Y$ is a map.
    Then $f(X)$ is connected.
\end{prop}
\begin{proof}
    Let $f(X)=(U\cap f(X))\cup(V\cap f(X))$ for open sets $U,V$ in $Y$.
    Then $X=f^{-1}(U)\cup f^{-1}(V)$, and $f^{-1}(U),f^{-1}(V)$ are open.
    Since $X$ is connected, one of $f^{-1}(U),f^{-1}(V)$ is empty, say $f^{-1}(V)$.
    Then $f(f^{-1}(V))=(V\cap f(X))$ is empty, hence $f(X)$ is connected.
\end{proof}

Another type of connectedness is path connected.
This is easier to describe.

\begin{defn}
    A \emph{path} in space $X$ is a continuous mapping from a closed interval to $X$.
\end{defn}

\begin{defn}
    A topological space is \emph{path connected} if any two points in it can be joined with a path.
    A subset of a topological space is \emph{path connected} if it is path connected with respective to subspace topology.
\end{defn}

\begin{lem}\label{lem:path connected}
    Path connected spaces are connected.
\end{lem}
\begin{proof}
    Suppose not.
    Let $X$ be a path connected but disconnected space, then $X=U\cup V$ for some nonempty disjoint open sets.
    Choose $x\in U,y\in V$, join $x,y$ with a path $f:I\to X$, $I$ is a closed interval.
    Then $f(I)$ is connected by \Cref{prop:connected image}, but $f(I)=(f(I)\cap U)\cup(f(I)\cap V)$, which indicates $f(I)$ is disconnected, contradiction.
\end{proof}

\begin{rem}
    \Cref{lem:path connected} shows that path connectedness is stronger than connectedness.
    For an example showing these two notions are not equivalent, please search ``topologist's sine curve''.
\end{rem}

\begin{prop}
    Let $X,Y$ be topological spaces and $X$ is path connected, $f:X\to Y$ is a map.
    Then $f(X)$ is path connected.
\end{prop}
\begin{proof}
    Let $y,y'\in f(X)$ with $f(x)=y$, $f(x')=y'$, and path $i:I\to X$ joins $x,x'$.
    Then $f\circ i$ is a path joining $y$ and $y'$.
\end{proof}

\begin{defn}
    Let $X$ be a topological space.
    Define an equivalence relation $\sim$ on $X$ as follows:
    two points $x\sim y$ if $x$ and $y$ can be joined with a path (please verify this is an equivalence relation).
    Then an element in $X/\sim$ is called a \emph{path connected component}.
\end{defn}

I think this is enough to be a rapid introduction to the language of general topology.